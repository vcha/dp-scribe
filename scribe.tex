\documentclass[twoside]{article}
\usepackage{amsmath}
\usepackage{amsfonts}
\usepackage{graphicx}
\usepackage{color}

\setlength{\oddsidemargin}{0.25 in}
\setlength{\evensidemargin}{-0.25 in}
\setlength{\topmargin}{-0.6 in}
\setlength{\textwidth}{6.5 in}
\setlength{\textheight}{8.5 in}
\setlength{\headsep}{0.75 in}
\setlength{\parindent}{0 in}
\setlength{\parskip}{0.1 in}

\newcommand{\lecture}[3]{
   \pagestyle{myheadings}
   \thispagestyle{plain}
   \newpage
   \setcounter{page}{1}
   \noindent
   \begin{center}
   \framebox{
      \vbox{\vspace{2mm}
    \hbox to 6.28in {{\bf 10-708:~Probabilistic Graphical Models
10-708, Spring 2012 \hfill}}
       \vspace{6mm}
       \hbox to 6.28in {{\Large \hfill #1  \hfill}}
       \vspace{6mm}
       \hbox to 6.28in {{\it Lecturer: #2 \hfill Scribes: #3}}
      \vspace{2mm}}
   }
   \end{center}
   \markboth{#1}{#1}
   \vspace*{4mm}
}

\newcommand{\Dir}{\mathrm{Dir}}
\newcommand{\Gam}{\mathrm{Gamma}}
\newcommand{\DP}{\mathrm{DP}}
\newcommand{\todo}[1]{{\color{red} TODO: #1}}

\begin{document}

\lecture{Bayesian nonparametrics}{Sinead Williamson}{Victor Chahuneau, Rory Donovan, Abhimanu Kumar}

\section{Introduction}
Nonparametric models allow the number of parameters to grow with sample size.

\section{Nonparametric regression with Gaussian processes}
Usually, polynomial regression is used to model a set of points. Unfortunately, as the numbers of degree of freedom include, this produces overfitting.

\paragraph{The Gaussian distribution}
Two important properties: if $y=[y_1,y_2]$ is Gaussian, then:
\begin{itemize}
    \item the marginal $y_1$ is Gaussian
    \item the conditional $y_1 \mid y_2$ is Gaussian
\end{itemize}

\paragraph{The Gaussian process} A GP corresponds to a generalization of a $N$-variate Gaussian as $N \longrightarrow \infty$, such that the marginals are still normally distributed. It is thus a distribution over functions controlled by a covariance function.
Typically, an exponential kernel is used to define the covariance function.

When observing data, the predictions of the model are close to the existing data points and generic wherever data is absent \todo{add figure from slides}.

\section{Nonparametric mixture models}
\subsection{Finite mixture models}
Bayesian mixture models are obtained by putting a prior and integrating out.

\subsection{The Dirichlet distribution}
A $K$-dimensional Dirichlet distribution, denoted $\Dir(\alpha)$, is parametrized b $K$ real numbers $\alpha_1 \dots \alpha_K$ which control where the probability mass is distributed over the $K$-dimensional simplex.
This can be seen by taking the excpectation of $\pi \sim \Dir(\alpha)$:
$$ \mathbb{E}(\pi_1 \dots \pi_K) = \frac{(\alpha_1 \dots \alpha_K)}{\sum_k \alpha_k} $$ 

An important property of the Dirichlet distribution is its conjugacy to the multinomial distribution. This derives from:
$$ p(\pi \mid x_1 \dots x_N) = \dots $$
In practice, this can be interpreted as smoothing the parameters.

The Dirichlet distribution is a distribution over distributions, defined by positive vectors which sum to one. A sample from a Dirichlet distribution $\pi \sim \Dir(\alpha)$ is itself a distribution over the parameter space (e.g. $\theta = (\mu, \Sigma) \in \mathbb{R}^n \times \mathbb{R}^{2n}$ for an $n$-variate Gaussian).

$$ p(x \mid \pi) = \sum_{k=1}^K \pi_k p(x \mid \theta_K) $$

\paragraph{Relation to the gamma distribution}

If $\nu_k \sim \Gam(\alpha_k, 1)$, then $\frac{\nu}{\sum_k \nu_k} \sim \Dir(\alpha)$.

\paragraph{Properties}

From the properties of the gamma distribution, we deduce that a $(K-1)$-dimensional Dirichlet distribution can be obtained from a $K$-dimensional Dirichlet distribution by summing the two first components.
\todo{renormalization}

\subsection{The Dirichlet process}
We can now generalize to an infinite number of parameters by splitting and renormalizing: $\pi \sim \Dir(\frac{\alpha}{K}), K \longrightarrow \infty$. This defines the Dirichlet process, which we write $G \sim \DP(\alpha, H)$. Now the combination can be written:

$$ p(x \mid G) = \sum_{k=1}^K \pi_k p(x \mid \theta_K) $$

The concentration parameter $\alpha$ controls the sparsity of the distribution, and the base measure $H$ determines the locations of the atoms $\delta_{\theta_k}$.

Another definition of the DP can be obtained by considering partitions of the probability space: \todo{A Dirichlet process is the unique distribution over probability distributions on some space Ω, such that for any finite partition [etc.]}

The posterior over partitions given an observation $x$ is a finite dimensional Dirichlet :
$ G \mid x \sim \DP(\alpha+1, \frac{\alpha H + \delta_x}{\alpha + 1})$

\todo{Finish posterior distribution}

\end{document}
